\documentclass[a4paper,12pt]{article}

\usepackage[T2A]{fontenc}	
\usepackage[utf8]{inputenc}	
\usepackage[english,russian]{babel}
\usepackage{amsmath,amsfonts,amssymb,amsthm,mathtools} 
\usepackage[pdftex]{graphicx}
\graphicspath{{images/}}
\usepackage{wrapfig}
\usepackage{wasysym}
\usepackage{textcomp}
\usepackage{indentfirst}
\usepackage{amsmath}
\usepackage[colorlinks,urlcolor=blue]{hyperref} %ссылки

\author{}
\title{
Горизонты физики.
Введение в фундаментальную фотонику и квантовую физику.
Лекция 2.
Введение в волновую оптику.
}
\date{\today}



\begin{document}

	\maketitle

	\newpage
	
	\section{Явления волновой оптики}
		\subsection{Поляризация}
		Для общего развития. В дальнейших расчётах использоваться не будет.
		
		Рассмотрим волновое уравнение:
		$$ E = E_0cos(\omega t - kz + \varphi_0), $$ где $\varphi_0$ --- начальная фаза
		
		$$ k = \frac{2 \pi}{\lambda} $$ --- волновое число
		
		Для неполяризованного света:		
		
		\begin{equation*}
		\begin{cases}
   			E_x = E_0cos(\omega t - kz + \varphi_0)\\
   			E_y = E_0cos(\omega t - kz + \varphi_0)
 		\end{cases}
		\end{equation*}
		
		Поляризованный же свет колеблется только по определённым осям.
		
		\vspace{0.5cm}
		\textbf{Линейная поляризация:}
		
		Колебания света происходят только по одной из осей. Соответственно, например, для колебаний по некоторой оси Ox:
		
		$$ E = E_x = E_0cos(\omega t - kz + \varphi_0) $$
		
		\vspace{0.5cm}
		\textbf{Нелинейная поляризация:}
		
		Для начала получается линейная поляризация, затем достигается относительный сдвиг волн по фазе.
		
		\vspace{0.5cm}
		\textbf{Поляроиды:}
		
		\textbf{TODO}
		
		\subsection{Интерференция}
		Явление взаимодействия нескольких волн.
		
		$$ \vec{E} = \vec{E_1} + \vec{E_2} $$
		$$ I \sim <|E^2|> = <|\vec{E_1} + \vec{E_2}|^2> = E_1^2+E_2^2+2(E_1, \, E_2) = I_1 + I_2 + 2(E_{01}, \, E_{02}) = $$
		$$ = I_1 + I_2 + 2\sqrt{I_1I_2}cos(k\cdot \Delta),$$
		где $ \Delta = n \cdot \Delta x$
		
		\vspace{0.5cm}
		Условия:
		
		$$ 1) \; (E_{01}, \, E_{02}) \neq 0 $$
		$$ 2) \;  \omega_1 = \omega_2 => k_1=k_2=\frac{\omega}{c}$$
		
		\vspace{0.5cm}
		Условие максимума интерференции:
		$$ \Delta_{max} = m\lambda, $$
		m - целое.
		
		\vspace{0.5cm}
		\textbf{Схема Юнга с щелями}
		
		Универсальная схема получения интерференции от нелазерного источника.
		
		Из геометрии при малом угле $\varphi$ :
		$$ \frac{x\cdot d}{L} = d \, tg \varphi, $$
		L --- расстояние от щели до экрана, d --- высота щели, x --- высота рассматриваемой точки
		
		Условие максимума интерференции:
		$$ d \, tg \varphi = m\cdot \lambda $$
		
		\vspace{0.5cm}
		\textbf{Схема Юнга с тремя щелями}
		Векторные диаграммы
		
		\vspace{0.5cm}
		\textbf{Дифракционная решётка}
		
		$N >> 1$, N --- число щелей.
		
		Дифракционная картина при этом содержит чёткие узкие пики и тусклый фон между ними для каждой длины волны (грубо говоря, его отсутствие).
		
		$$ dsin \varphi = m\lambda < d $$
		$$ m_{max} \leq \frac{d}{\lambda} $$
		
		\vspace{0.5cm}
		Различимость пиков разных частот для дифракционной решётки, если считать различимыми длины волн, отличающиеся хотя бы на $20 \, \%$:
		$$ \delta \lambda = \frac{\lambda}{m\cdot N} $$
		
		
		\vspace{0.5cm}
		Боковые максимумы менее яркие, чем центральные. Этот эффект возникает по причине неточечности щели.
		
		\vspace{0.5cm}
		\textbf{Область дисперсии}
		
		Свободную область дисперсии определяет максимальную ширину спектрального интервала $\delta \lambda$ исследуемого излучения, при которой спектры соседних порядков еще не перекрываются. Длинноволновый конец спектра m-того порядка совпадает с коротковолновым концом спектра (m+1) порядка при выполнении условия
		$$ m ( \lambda + \delta \lambda ) = (m+1)\lambda $$
		К примеру,
		$$ m \lambda_{red} = (m+1)\lambda_{violet} $$

		
		\subsection{Дифракция}		
		Принцип Бабине
		
		\textbf{TODO}
	
\end{document}